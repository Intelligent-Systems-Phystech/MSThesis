\documentclass[12pt,oneside]{article}
\usepackage[russian,english]{babel}
\usepackage[utf8]{inputenc}
\usepackage{abstract}
\usepackage{amsmath,amssymb,mathrsfs,mathtext,amsthm}
\usepackage{a4wide}
\usepackage[T2A]{fontenc}
\usepackage{subfig}
\usepackage{url}
\usepackage[usenames]{color}
\usepackage{colortbl}

\newcommand{\hdir}{.}
\usepackage{hyperref}       % clickable links
\usepackage{lineno}
\usepackage{graphicx,multicol}
\usepackage{epstopdf}
\usepackage{cite}
\usepackage{amsmath,amssymb,mathrsfs,mathtext}
\usepackage{tikz}
\usetikzlibrary{shapes,arrows,shadows}

\newtheorem{theorem}{Theorem}
\newtheorem{proposition}{Proposition}

\theoremstyle{definition}
\newtheorem{definition}{Definition}

\usepackage{algorithm}
\usepackage[noend]{algcompatible}

\usepackage{multirow}

\usepackage{caption}
\usepackage{geometry}
\usepackage{setspace}
%\onehalfspacing
\geometry{verbose,a4paper,tmargin=2cm,bmargin=2cm,lmargin=3cm,rmargin=1.5cm}

\begin{document}
	\thispagestyle{empty}
	\begin{figure}
		\centering
		\includegraphics[width=0.38\linewidth]{sk_logo}
	\end{figure}
	\noindent {\bf University:} Skolkovo Institute of Science and Technology (Skoltech) \\
	{\bf Student Full Name:} Roman Isachenko \\
	{\bf Master Thesis Topic:} Dimensionality reduction for signal analysis \\


The goal of this Master Thesis is to investigate the dependencies in input and target spaces, as well as to build a stable model for time series decoding in the case of correlated data description.
The model should adequately describe both input and target spaces with high dimensions.
To take into account the dependencies simultaneously the author proposesto reduce the dimensionality using latent space.
The feature selection method for multidimensional regression problem are proposed.
A combination of feature selection and dimensionality reduction is investigated.
The model for signal prediction in the high-dimensional space is introduced.
The proposed feature selection algorithms deliver stable and adequate solutions in correlated spaces.

The Master Thesis consists of 6 chapters.
The introduction provides a literature review and a brief description of the study.
The second section is devoted to the problem of reducing the space dimension.
The third section describes the PLS regression with proof of its optimality.
In section 4 the feature selection algorithms are proposed for the problem of multivariate regression. 
The statements with the justification of the proposed algorithms are proved.
Section 5 is devoted to the computational experiments with analysis of results and discussion of algorithm properties.
In conclusion the results of the study and the author's contribution are described.
The experiments were conducted correctly, the conclusions were competently formulated.

The main disadvantage of the work is the study restriction to the linear case. Further development of the work may be aimed at consideration of nonlinear models.

The study is an actual research, the reviewed work meets the requirements for the diploma works at the Skoltech. 
Presented Master’s Thesis may be considered as complete qualification paper for a Master’s student. \\[4mm]
{\bf Recommended grade for the Master’s Thesis is A}\\[4mm]
Reviewer status:  co-advisor \\[2mm]
Full Name: Vadim Strijov\\[2mm]
University/Organization: Moscow Institute of Physics and Technology\\[4mm]
Signature: \hspace{7cm}Date:

\end{document}

\documentclass[12pt,oneside]{article}
\usepackage[russian]{babel}
\usepackage[utf8]{inputenc}
\usepackage{abstract}
\usepackage{amsmath,amssymb,mathrsfs,mathtext,amsthm}
\usepackage{a4wide}
\usepackage[T2A]{fontenc}
\usepackage{subfig}
\usepackage{url}
\usepackage[usenames]{color}
\usepackage{colortbl}

\newcommand{\hdir}{.}
\usepackage{hyperref}       % clickable links
\usepackage{lineno}
\usepackage{graphicx,multicol}
\usepackage{epstopdf}
\usepackage{cite}
\usepackage{amsmath,amssymb,mathrsfs,mathtext}
\usepackage{tikz}
\usetikzlibrary{shapes,arrows,shadows}

\newtheorem{theorem}{Theorem}
\newtheorem{proposition}{Proposition}

\theoremstyle{definition}
\newtheorem{definition}{Definition}

\usepackage{algorithm}
\usepackage[noend]{algcompatible}

\usepackage{multirow}

\usepackage{caption}
\usepackage{geometry}
\usepackage{setspace}
%\onehalfspacing
\geometry{verbose,a4paper,tmargin=2cm,bmargin=2cm,lmargin=3cm,rmargin=1.5cm}

\begin{document}
	\thispagestyle{empty}
	\begin{center}
		\sc
		МОСКОВСКИЙ ФИЗИКО-ТЕХНИЧЕСКИЙ ИНСТИТУТ \\
		(ГОСУДАРСТВЕННЫЙ УНИВЕРСИТУТ) \\
		ФАКУЛЬТЕТ УПРАВЛЕНИЯ И ПРИКЛАДНОЙ МАТЕМАТИКИ \\
		ВЫЧИСЛИТЕЛЬНЫЙ ЦЕНТР ИМ. А. А. ДОРОДНИЦЫНА РАН\\[1cm]
	\end{center}
\begin{center}
	{\bf
	Рецензия на магистерскую диссертацию \\[2mm]
	Снижение размерности пространства в задачах анализа сигналов} \\ [2mm]
	студента 6 курса \\
	Исаченко Романа Владимировича
\end{center}

Магистерская диссертация посвящена задаче снижения размерности в пространствах коррелированных признаковых описаний. 
Задача состоит в том, чтобы учесть зависимости как в пространстве объектов, так и в пространстве ответов.
Предложены алгоритмы, обобщающие существующие методы выбора признаков на многомерный случай.
В качестве прикладной области рассматривается задача моделирования нейрокомпьютерного интерфейса.
В эксперименте исследуются свойства предложенных алгоритмов.

Магистерская диссертация представляет собой документ из 33 страниц, состоящий из 6 глав.
Первая глава представляет собой введение с изложенной мотивацией исследования, обзором литературы и описанием текущего состояния области. 
В разделе 2 приводятся постановки задач многомерной регрессии, снижения размерности пространства и выбора признаков. 
В разделе 3 подробно описан используемый алгоритм частных наименьших квадратов. 
Раздел 4 содержит предлагаемые алгоритмы выбора признаков в случае задачи многомерной регрессии. 
Каждый подраздел посвящен отдельному методу с постановкой задачи и описанием свойств метода. 
Доказаны утверждения с обоснованием предложенных алгоритмов.
Раздел 5 посвящен вычислительным экспериментам с анализом и обсуждением полученных результатов.
Последний раздел завершает исследование.

Недостатком диссертации является рассмотрение исключительно линейных моделей. 
Возможным развитием исследования является рассмотрение нелинейных моделей. 
Кроме того, вычислительный эксперимент было бы предпочтительнее провести в приложении к другой прикладной области. 

Несмотря на отмеченные недостатки, рассмотренная работа соответствует требованиям к дипломным работам МФТИ.
Представленная магистерская диссертация рассматривается как полная квалификационная работа для магистранта.
Работа заслуживает оценки <<отлично>>, а Р.В. Исаченко~-- присвоения квалификации магистра.
\\[4mm]
{\bf Рецензент:} \\
Дьяконов Александр Геннадьевич \\
Доктор физико-математических наук \\
Профессор Московского государственного университета \\[4mm]
Подпись: \hspace{7cm}Дата:


\end{document}

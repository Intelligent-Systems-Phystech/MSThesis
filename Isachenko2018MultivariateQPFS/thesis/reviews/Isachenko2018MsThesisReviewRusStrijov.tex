\documentclass[12pt,oneside]{article}
\usepackage[russian]{babel}
\usepackage[utf8]{inputenc}
\usepackage{abstract}
\usepackage{amsmath,amssymb,mathrsfs,mathtext,amsthm}
\usepackage{a4wide}
\usepackage[T2A]{fontenc}
\usepackage{subfig}
\usepackage{url}
\usepackage[usenames]{color}
\usepackage{colortbl}

\newcommand{\hdir}{.}
\usepackage{hyperref}       % clickable links
\usepackage{lineno}
\usepackage{graphicx,multicol}
\usepackage{epstopdf}
\usepackage{cite}
\usepackage{amsmath,amssymb,mathrsfs,mathtext}
\usepackage{tikz}
\usetikzlibrary{shapes,arrows,shadows}

\newtheorem{theorem}{Theorem}
\newtheorem{proposition}{Proposition}

\theoremstyle{definition}
\newtheorem{definition}{Definition}

\usepackage{algorithm}
\usepackage[noend]{algcompatible}

\usepackage{multirow}

\usepackage{caption}
\usepackage{geometry}
\usepackage{setspace}
%\onehalfspacing
\geometry{verbose,a4paper,tmargin=2cm,bmargin=2cm,lmargin=3cm,rmargin=1.5cm}

\begin{document}
	\thispagestyle{empty}
	\begin{center}
		\sc
		МОСКОВСКИЙ ФИЗИКО-ТЕХНИЧЕСКИЙ ИНСТИТУТ \\
		(ГОСУДАРСТВЕННЫЙ УНИВЕРСИТУТ) \\
		ФАКУЛЬТЕТ УПРАВЛЕНИЯ И ПРИКЛАДНОЙ МАТЕМАТИКИ \\
		ВЫЧИСЛИТЕЛЬНЫЙ ЦЕНТР ИМ. А. А. ДОРОДНИЦЫНА РАН\\[1cm]
	\end{center}
\begin{center}
	{\bf
	Рецензия на магистерскую диссертацию \\[2mm]
	Снижение размерности пространства в задачах анализа сигналов} \\ [2mm]
	студента 6 курса \\
	Исаченко Романа Владимировича
\end{center}

Цель магистерской диссертации состоит в исследовании зависимости в пространствах объектов и ответов, а также в построении устойчивой модели декодирования временных рядов в случае коррелированного описания данных.
Требуется построить модель, адекватно описывающую как пространство объектов так и пространство ответов при наблюдаемой мультикорреляции в обоих пространствах высокой размерности.
Для учёта зависимостей одновременно в пространствах объектов и ответов предлагается снизить размерность с использованием скрытого пространства.
Предложены методы для выбора признаков для задачи многомерной регрессии.
Предложена комбинация методов выбора признаков и снижения размерности пространства.
Создан макет системы, прогнозирующей сигналы в пространстве большой размерности.
Предложенные алгоритма выбора признаков доставляют устойчивые и адекватные решения в коррелированных пространствах высокой размерности.

Магистерская диссертация состоит из 6 глав.
Во введении представлен обзор литературы и краткое описание исследования.
Второй раздел посвящен постановке задачи снижения размерности пространства.
Третий раздел описывает алгоритм PLS регрессии с доказательством его оптимальности.
В разделе 4 предлагаются алгоритмы выбора признаков для задачи многомерной регрессии. 
Доказаны утверждения с обоснованием предложенных алгоритмов.
Раздел 5 посвящен вычислительным экспериментам с анализом результатов и обсуждением свойств алгоритмов.
В заключении подведён итог исследования и описан вклад автора.
Все эксперименты проведены корректно, выводы грамотно сформулированы.

К недостаткам работы можно отнести ограничение задачи на линейный случай. Дальнейшее развитие работы может быть нацелено на рассмотрение нелинейных моделей.

Работа является актуальным исследованием, удовлетворяет требованиям, предъявляемым к магистерским диссертациям в МФТИ и заслуживает оценки <<отлично>>, а Р.В. Исаченко~--присвоения квалификации магистра.
\\[4mm]
{\bf Рецензент:} \\
Стрижов Вадим Викторович \\
Доктор физико-математических наук \\
Профессор Московского физико-технического института \\[4mm]
Подпись: \hspace{7cm}Дата:


\end{document}

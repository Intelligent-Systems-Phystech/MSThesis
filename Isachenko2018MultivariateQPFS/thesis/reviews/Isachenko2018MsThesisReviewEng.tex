\documentclass[12pt,oneside]{article}
\usepackage[russian,english]{babel}
\usepackage[utf8]{inputenc}
\usepackage{abstract}
\usepackage{amsmath,amssymb,mathrsfs,mathtext,amsthm}
\usepackage{a4wide}
\usepackage[T2A]{fontenc}
\usepackage{subfig}
\usepackage{url}
\usepackage[usenames]{color}
\usepackage{colortbl}

\newcommand{\hdir}{.}
\usepackage{hyperref}       % clickable links
\usepackage{lineno}
\usepackage{graphicx,multicol}
\usepackage{epstopdf}
\usepackage{cite}
\usepackage{amsmath,amssymb,mathrsfs,mathtext}
\usepackage{tikz}
\usetikzlibrary{shapes,arrows,shadows}

\newtheorem{theorem}{Theorem}
\newtheorem{proposition}{Proposition}

\theoremstyle{definition}
\newtheorem{definition}{Definition}

\usepackage{algorithm}
\usepackage[noend]{algcompatible}

\usepackage{multirow}

\usepackage{caption}
\usepackage{geometry}
\usepackage{setspace}
%\onehalfspacing
\geometry{verbose,a4paper,tmargin=2cm,bmargin=2cm,lmargin=3cm,rmargin=1.5cm}

\begin{document}
	\thispagestyle{empty}
	\begin{figure}
		\centering
		\includegraphics[width=0.38\linewidth]{sk_logo}
	\end{figure}
	\noindent {\bf University:} Skolkovo Institute of Science and Technology (Skoltech) \\
	{\bf Student Full Name:} Roman Isachenko \\
	{\bf Master Thesis Topic:} Dimensionality reduction for signal analysis \\

The Master Thesis considers the dimensionality reduction problem in the case of correlated data representation. 
The challenge is to incorporate the dependencies in both input and target spaces.
The author proposes the algorithms to extend the existing feature selection methods to the multivariate case.
The Brain Computer Interface problem is investigated as an potential application of research. 
The experiment compares the proposed algorithms with the existing ones.

The presented Master Thesis is a 33 page document, consisting of 6 chapters.
First chapter is an introduction with explained research motivation, literature review and description of current state of the field. 
Section 2 states the problems of multivariate regression, dimensionality reduction and feature selection. 
In Section 3 the PLS algorithm is described in details. 
Section 4 contains the proposed new algorithms for the multivariate feature selection. 
Each subsection is devoted to the particular method with the problem statement and the description of method properties. 
The statements with justification for the proposed algorithms are proven.
Section 5 is devoted to the computational experiments with the comprehensive analysis and discussion of the results.
The last section concludes the research.

The drawback of the thesis is consideration of only linear models. 
The possible development of the research is an investigation of non-linear models. 
Additionally, the computational experiment it would be preferable to carry out with the different application for proposed algorithms. 

Despite the noted shortcomings, the reviewed work meets the requirements for the diploma works at the Skoltech.
Presented Master’s Thesis may be considered as complete qualification paper for a Master’s student. \\[4mm]
{\bf Recommended grade for the Master’s Thesis is A}\\[4mm]
Reviewer status:    EXTERNAL REVIEWER     or     RESEARCH ADVISOR\\[2mm]
Full Name: Alexander Dyakonov\\[2mm]
University/Organization: Moscow State University\\[4mm]
Signature: \hspace{7cm}Date:

\end{document}

\documentclass[12pt,oneside]{article}
\usepackage[russian,english]{babel}
\usepackage[utf8]{inputenc}
\usepackage{abstract}
\usepackage{amsmath,amssymb,mathrsfs,mathtext,amsthm}
\usepackage{a4wide}
\usepackage[T2A]{fontenc}
\usepackage{subfig}
\usepackage{url}
\usepackage[usenames]{color}
\usepackage{colortbl}

\newcommand{\hdir}{.}
\usepackage{hyperref}       % clickable links
\usepackage{lineno}
\usepackage{graphicx,multicol}
\usepackage{epstopdf}
\usepackage{cite}
\usepackage{amsmath,amssymb,mathrsfs,mathtext}
\usepackage{tikz}
\usetikzlibrary{shapes,arrows,shadows}

\newtheorem{theorem}{Theorem}
\newtheorem{proposition}{Proposition}

\theoremstyle{definition}
\newtheorem{definition}{Definition}

\usepackage{algorithm}
\usepackage[noend]{algcompatible}

\usepackage{multirow}

\usepackage{caption}

\usepackage{geometry}
\usepackage{setspace}
%\onehalfspacing
\geometry{verbose,a4paper,tmargin=2.5cm,bmargin=2.5cm,lmargin=1.2cm,rmargin=1.2cm}

\usepackage[center]{titlesec}
\renewcommand{\refname}{REFERENCES}
% \renewcommand{\contentsname}{REFERENCES}
\titlelabel{\thetitle.\quad}
\newcommand{\sectionbreak}{\clearpage}
\let\originalparagraph\paragraph
\renewcommand{\paragraph}[2][.]{\originalparagraph{#2#1}}
\titleformat{\section}[block]{\Large\bfseries\filcenter}{\thesection.\;}{0.1em}{\uppercase}

\titleformat{\subsection}[block]{\large\bfseries\filcenter}{\thesubsection.\;}{0.1em}{}
\titleformat{\subsubsection}[block]{\bfseries\filcenter}{\thesubsubsection.\;}{0.1em}{} 

\newcommand\undermat[2]{%
	\makebox[0pt][l]{$\smash{\underbrace{\phantom{%
					\begin{matrix}#2\end{matrix}}}_{\text{$#1$}}}$}#2}
\usepackage{tikz}
\usepackage{tikzpagenodes}

\newcommand{\borders}{\tikz[remember picture,overlay] \draw []
	(current page text area.south west)
	rectangle
	(current page text area.north east)
	;}

\begin{document}
	\borders
	\pagestyle{empty}
	
	\begin{figure}
		\vspace{1.7cm}
		\centering
		\includegraphics[width=0.38\linewidth]{sk_logo}
	\end{figure}
	\begin{center}
		\sc
		MASTER'S THESIS\\[5mm]
		\bf{ \large
		Dimensionality Reduction in Signal Analysis}\\[25mm]
		\rm
		Master's Educational Program: <<Data Science>>\\[3mm]
	\end{center}
		\begin{minipage}{0.5\linewidth}
		\begin{flushright}
		Student: \hphantom{1}\\[9mm]
		Reseach Advisor: \hphantom{1}\\[9mm]
		Co-Advisor: \hphantom{1}\\
	\end{flushright}
	\end{minipage}%
	\begin{minipage}{0.48\linewidth}
			\vspace{.55cm}
			\begin{flushleft}
			Roman Isachenko\\[9mm]
			Maxim Fedorov, Full professor, \\[9mm]
			Vadim Strijov, Research Scientist \\(Moscow Institute of Physics and Technology)
			\\
			\end{flushleft}
	\end{minipage}%

	\vfill
	\begin{center}
		Moscow, 2018 \\[9mm]
		\scriptsize{
		Copyright 2018 Author. All rights reserved. \\[7mm]
		
		The author hereby grants to Skoltech permission to reproduce and to distribute publicly paper and electronic copies \\ 
		of this thesis document in whole and in part in any medium now known or hereafter created.\\[15mm]}
	\end{center}

	\newpage
	\borders
	\pagestyle{empty}
	
	\begin{figure}
		\vspace{1.7cm}
		\centering
		\includegraphics[width=0.38\linewidth]{sk_logo}
	\end{figure}
	\begin{center}
		\sc 
			МАГИСТЕРСКАЯ ДИССЕРТАЦИЯ\\[5mm]
		\bf {\large
		Снижение размерности пространства \\в задачах анализа сигналов}\\[19mm]
		\rm
		Магистерская образовательная программа: <<Наука о данных>>\\[3mm]
	\end{center}
	\begin{minipage}{0.5\linewidth}
		\begin{flushright}
			Студент: \hphantom{1}\\[9mm]
			Научный руководитель: \hphantom{1}\\[9mm]
			Со-руководитель: \hphantom{1}\\
		\end{flushright}
	\end{minipage}%
	\begin{minipage}{0.48\linewidth}
		\vspace{.55cm}
		\begin{flushleft}
			Исаченко Роман Владимирович\\[9mm]
			Федоров М.В., Профессор, \\[9mm]
			Стрижов В.В., Научный сотрудник (Московский физико-технический институт)\\
		\end{flushleft}
	\end{minipage}%
	
	\vfill
	\begin{center}
		Москва, 2018 \\[9mm]
		\scriptsize{
			Авторское право 2018. Все права защищены. \\[7mm]
			
			Автор настоящим дает Сколковскому институту науки и технологий \\
			pазрешение на воспроизводство и свободное распространение бумажных и электронных копий настоящей \\ диссертации в целом или частично на любом ныне существующем или созданном в будущем носителе. \\[12mm]}
	\end{center}
\end{document}

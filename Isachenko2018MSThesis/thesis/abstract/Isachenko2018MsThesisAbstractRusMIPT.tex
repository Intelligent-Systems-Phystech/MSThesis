\documentclass[12pt,oneside]{article}
\usepackage[russian]{babel}
\usepackage[utf8]{inputenc}
\usepackage{abstract}
\usepackage{amsmath,amssymb,mathrsfs,mathtext,amsthm}
\usepackage{a4wide}
\usepackage[T2A]{fontenc}
\usepackage{subfig}
\usepackage{url}
\usepackage[usenames]{color}
\usepackage{colortbl}

\newcommand{\hdir}{.}
\usepackage{hyperref}       % clickable links
\usepackage{lineno}
\usepackage{graphicx,multicol}
\usepackage{epstopdf}
\usepackage{cite}
\usepackage{amsmath,amssymb,mathrsfs,mathtext}
\usepackage{tikz}
\usetikzlibrary{shapes,arrows,shadows}

\newtheorem{theorem}{Theorem}
\newtheorem{proposition}{Proposition}

\theoremstyle{definition}
\newtheorem{definition}{Definition}

\usepackage{algorithm}
\usepackage[noend]{algcompatible}

\usepackage{multirow}

\usepackage{caption}
\usepackage{geometry}
\usepackage{setspace}
%\onehalfspacing
\renewcommand{\baselinestretch}{1.5}
\geometry{verbose,a4paper,tmargin=2cm,bmargin=2cm,lmargin=3cm,rmargin=1.5cm}

\usepackage[center]{titlesec}
\renewcommand{\refname}{REFERENCES}
% \renewcommand{\contentsname}{REFERENCES}
\titlelabel{\thetitle.\quad}
\newcommand{\sectionbreak}{\clearpage}
\let\originalparagraph\paragraph
\renewcommand{\paragraph}[2][.]{\originalparagraph{#2#1}}
\titleformat{\section}[block]{\Large\bfseries\filcenter}{\thesection.\;}{0.1em}{\uppercase}

\titleformat{\subsection}[block]{\large\bfseries\filcenter}{\thesubsection.\;}{0.1em}{}
\titleformat{\subsubsection}[block]{\bfseries\filcenter}{\thesubsubsection.\;}{0.1em}{} 

\begin{document}

\setcounter{page}{2}
\begin{center}
	\textbf{АННОТАЦИЯ:} 
\end{center}

Данная работа посвящена задаче декодирования сигнала для построения нейрокомпьютерного интерфейса.
Нейрокомпьютерный интерфейс помогает людям с ограниченными возможностями восстановить их мобильность.
Целью исследования является построение модели, предсказывающей положение конечности по сигналам мозга. 
Проблема заключается в избыточности исходного описания данных. 
Корреляция измерений прибора приводит к корреляции во входном пространстве описаний модели. 
Кроме того, рассматривается многомерный случай, целевая переменная является вектором из последовательных положений руки в пространстве. 
Зависимость между последовательными позициями руки приводит к корреляциям в пространстве ответов.
Для устранения избыточной корреляции в признаковом описании объектов используются методы снижения размерности и выбора признаков.

Регрессия методом частных наименьших квадратов~(PLS) используется в качестве базовой модели для снижения размерности пространства.
Данная модель проецирует входные объекты и ответы в скрытое пространство и максимизирует ковариации между проекциями.
Сочетание зависимостей входных объектов и ответов позволяет построить устойчивую модель.

Снижение размерности не поозволяет построить разреженную модель. Разреженность достигается путем выбора признаков.
Большинство методов выбора признаков не используют зависимости в пространстве ответов.
В работе предлагается новый подход к выбору признаков в случае многомерной регрессии.
Для учета корреляций в матрице ответов предлагается обобщить идею алгоритма выбора признаков с помощью квадратичного программирования~(QPFS). 
Алгоритм QPFS выбирает некоррелированные объекты, которые релеванты столбцам матрицы ответов. 
Предлагаемые методы накладывают веса на столбцы матрицы ответов. 
Идея состоит в том, чтобы оштрафовать коррелированные столбцы и уменьшить их влияние на выбор признаков. 

Вычислительный эксперимент проводится на реальном наборе данных электрокортикограмм (ЭКОГ). 
Предложенные алгоритмы сравниваются по различным критериям, таким как стабильность и точность прогноза.
Алгоритмы показывают результаты выше базового алгоритма.
Сравнивается модель линейной регрессии с использованием QPFS алгоритма и модель регрессии частных наименьших квадратов.
Наилучший результат достигается комбинацией алгоритмов QPFS и PLS.
\end{document}

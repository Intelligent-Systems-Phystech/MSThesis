\documentclass[12pt,oneside]{article}
\usepackage[russian]{babel}
\usepackage[utf8]{inputenc}
\usepackage{abstract}
\usepackage{amsmath,amssymb,mathrsfs,mathtext,amsthm}
\usepackage{a4wide}
\usepackage[T2A]{fontenc}
\usepackage{subfig}
\usepackage{url}
\usepackage[usenames]{color}
\usepackage{colortbl}

\newcommand{\hdir}{.}
\usepackage{hyperref}       % clickable links
\usepackage{lineno}
\usepackage{graphicx,multicol}
\usepackage{epstopdf}
\usepackage{cite}
\usepackage{amsmath,amssymb,mathrsfs,mathtext}
\usepackage{tikz}
\usetikzlibrary{shapes,arrows,shadows}

\newtheorem{theorem}{Theorem}
\newtheorem{proposition}{Proposition}

\theoremstyle{definition}
\newtheorem{definition}{Definition}

\usepackage{algorithm}
\usepackage[noend]{algcompatible}

\usepackage{multirow}

\usepackage{caption}

\usepackage{geometry}
\usepackage{setspace}
%\onehalfspacing
\geometry{verbose,a4paper,tmargin=2.5cm,bmargin=2.5cm,lmargin=1.2cm,rmargin=1.2cm}

\usepackage[center]{titlesec}
\renewcommand{\refname}{REFERENCES}
% \renewcommand{\contentsname}{REFERENCES}
\titlelabel{\thetitle.\quad}
\newcommand{\sectionbreak}{\clearpage}
\let\originalparagraph\paragraph
\renewcommand{\paragraph}[2][.]{\originalparagraph{#2#1}}
\titleformat{\section}[block]{\Large\bfseries\filcenter}{\thesection.\;}{0.1em}{\uppercase}

\titleformat{\subsection}[block]{\large\bfseries\filcenter}{\thesubsection.\;}{0.1em}{}
\titleformat{\subsubsection}[block]{\bfseries\filcenter}{\thesubsubsection.\;}{0.1em}{} 

\newcommand\undermat[2]{%
	\makebox[0pt][l]{$\smash{\underbrace{\phantom{%
					\begin{matrix}#2\end{matrix}}}_{\text{$#1$}}}$}#2}
\usepackage{tikz}
\usepackage{tikzpagenodes}


\begin{document}

\begin{titlepage}
	\thispagestyle{empty}
	\begin{center}
		\sc
		
		Министерство науки и высшего образования \\ Российской Федерации \\
		"<Московский физико-технический институт \\
		(государственный университет)"> \\
		Физтех-школа прикладной математики и информатики\\
		Факультет управления и прикладной математики \\
		Кафедра <<Интеллектуальные системы>>\\[35mm]
		\rm\large
		Исаченко Роман Владимирович\\[10mm]
		\bf\Large
		Снижение размерности в задачах анализа сигналов\\[10mm]
		\rm\normalsize
		03.04.01 ---Прикладные математика и физика\\[10mm]
		\sc
		
		Выпускная квалификационная работа\\
		(магистерская диссертация)\\[10mm]
	\end{center}
	\hfill\parbox{80mm}{
		\begin{flushleft}
			\bf
			Научный руководитель:\\
			\rm
			д.~ф.-м.~н. Стрижов Вадим Викторович\\[5cm]
		\end{flushleft}
	}
	
	\vspace{\fill}
	\begin{center}
		Москва\\
		2018
	\end{center}
\end{titlepage}
\end{document}
